% $Header: /cvsroot/latex-beamer/latex-beamer/solutions/conference-talks/conference-ornate-20min.en.tex,v 1.6 2004/10/07 20:53:08 tantau Exp $
\pdfminorversion=4

\documentclass[10pt,xcolor=svgnames]{beamer}

\mode<presentation>
{
  %\usetheme{Madrid}
  \usetheme{Boadilla}
  % or ...

  \setbeamercovered{transparent}
  \setbeamertemplate{navigation symbols}{}
  % suppress slide number for continued slides
  \setbeamertemplate{frametitle continuation}[from second][]
  % or whatever (possibly just delete it)
}


\usepackage{graphicx}
\usepackage{aliases}
\usepackage{colortbl}
\usepackage{color}
\usepackage{stmaryrd}
\usepackage{listings}


\usepackage[english]{babel}
% or whatever

%\usepackage[latin1]{inputenc}
% or whatever

%\usepackage{times}
%\usepackage[T1]{fontenc}
% Or whatever. Note that the encoding and the font should match. If T1
% does not look nice, try deleting the line with the fontenc.

\usefonttheme{serif}
\setbeamertemplate{itemize item}[circle]
\setbeamertemplate{itemize subitem}[triangle]
\setbeamertemplate{enumerate item}[circle]
\setbeamertemplate{enumerate subitem}[square]

%\newcommand{\clr}[2]{{\color{#1}#2\color{black}}}
\newcommand{\hoz}{H^1_0}
\newcommand{\hozO}{H^1_0(\Omega)}
\newcommand{\hoO}{H^1(\Omega)}
\newcommand{\htO}{H^2(\Omega)}
\newcommand{\htzO}{H^2_0(\Omega)}

%\everydisplay{\color{blue}}


\title[1D BVP] % (optional, use only with long paper titles)
{Solving 1-D BVP}

%\subtitle
%{1-D boundary value problem}

\author[Praveen. C] % (optional, use only with lots of authors)
{Praveen. C\\
{\tt praveen@math.tifrbng.res.in}}
% - Give the names in the same order as the appear in the paper.
% - Use the \inst{?} command only if the authors have different
%   affiliation.

\institute[TIFR-CAM] % (optional, but mostly needed)
{
   \includegraphics[height=1.0cm]{tifr.png}\\
   Tata Institute of Fundamental Research\\
   Center for Applicable Mathematics\\
   Bangalore 560065\\
{\tt http://math.tifrbng.res.in}
}
% - Use the \inst command only if there are several affiliations.
% - Keep it simple, no one is interested in your street address.

%\date[ADA, 16 Jan 2010] % (optional, should be abbreviation of conference name)
%{MAT2010\\
%Aeronautical Development Agency, Bangalore\\
%16 January, 2010}
% - Either use conference name or its abbreviation.
% - Not really informative to the audience, more for people (including
%   yourself) who are reading the slides online

%\subject{Computational Fluid Dynamics}
% This is only inserted into the PDF information catalog. Can be left
% out. 

% If you have a file called "university-logo-filename.xxx", where xxx
% is a graphic format that can be processed by latex or pdflatex,
% resp., then you can add a logo as follows:

%\pgfdeclareimage[height=0.5cm]{university-logo}{inria}
%\logo{\pgfuseimage{university-logo}}

% Delete this, if you do not want the table of contents to pop up at
% the beginning of each subsection:
%\AtBeginSubsection[]
%{
%  \begin{frame}<beamer>
%    \frametitle{Outline}
%    \tableofcontents[currentsection,currentsubsection]
%  \end{frame}
%}

% If you wish to uncover everything in a step-wise fashion, uncomment
% the following command: 

%\beamerdefaultoverlayspecification{<+->}

\begin{document}

\lstset{
	language=C++,
	keywordstyle=\bfseries\ttfamily\color[rgb]{0,0,1},
	identifierstyle=\ttfamily,
	commentstyle=\color[rgb]{0.133,0.545,0.133},
	stringstyle=\ttfamily\color[rgb]{0.627,0.126,0.941},
	showstringspaces=false,
	basicstyle=\small,
	numberstyle=\footnotesize,
	numbers=left,
	stepnumber=1,
	numbersep=10pt,
	tabsize=2,
	breaklines=true,
	prebreak = \raisebox{0ex}[0ex][0ex]{\ensuremath{\hookleftarrow}},
	breakatwhitespace=false,
	aboveskip={0.1\baselineskip},
    columns=fixed,
    upquote=true,
    extendedchars=true,
% frame=single,
    backgroundcolor=\color[rgb]{0.9,0.9,0.9}
}

\begin{frame}
  \titlepage
\end{frame}

%\begin{frame}
% \frametitle{Outline}
%\tableofcontents
% You might wish to add the option [pausesections]
%\end{frame}
%#############################################################################
\begin{frame}[allowframebreaks]
\frametitle{1-D BVP}
\[
-u'' = f \quad x \in (0,1)
\]
\[
u(0) = u_a, \qquad u(1) = u_b
\]
Grid with $n$ points, $h = \frac{1}{n-1}$, $x_j = j h$
\[
0 = x_0 < x_1 < \ldots < x_{n-2} < x_{n-1} = 1
\]
Finite difference
\[
- \frac{u_{j-1} - 2 u_j + u_{j+1}}{h^2} = f_j, \quad 1 \le j \le n-1
\]
or with $a = \frac{2}{h^2}$, $b = -\frac{1}{h^2}$
\begin{eqnarray*}
u_0 &=& u_a \\
b u_{j-1} + a u_j + b u_{j+1} &=& f_j, \quad 1 \le j \le n-1 \\
u_{n-1} &=& u_b
\end{eqnarray*}
Matrix is non-symmetric. Make it symmetric by eliminating $u_0$, $u_{n-1}$ from all other equations.
\begin{eqnarray*}
a u_0 &=& a u_a \\
 a u_1 + b u_{2} &=& f_1 - b u_a \\
b u_{j-1} + a u_j + b u_{j+1} &=& f_j, \qquad\qquad 2 \le j \le n-2 \\
b u_{n-3} + a u_{n-2} &=& f_{n-1} - b u_b \\
a u_{n-1} &=& a u_b
\end{eqnarray*}
\end{frame}
%-----------------------------------------------------------------------------------

\end{document}